\chapter{Performance}\label{sec:performance}

There exist very many programs, some of which will profit greatly from our rewrite rules and some of which might not differ at all.
This makes evaluating the performance of our rewrite rules in a general case very hard, so instead we will again look at the shift example discussed earlier and see how it fares, to get an idea of the possible performance improvements.
This way we can still get an idea of the potential benefits of the rewrite rules, using a realistic example.

Figure~\ref{fig:performance} below shows the performance of both the original and rewritten program when shifting a list of length $20000$ by some amount to the right.
\begin{figure}[!h]
    \centering
    \caption{Running time of shifting a list of length 20000}
    \pgfplotsset{scaled x ticks=false}
    \begin{tikzpicture}
        \begin{axis}[
                xlabel=Shift distance,
                ylabel=Time (ms),
                xmin=0, xmax=10000,
                ymin=0, ymax=400,
                legend pos=south east,
                ymajorgrids=true,
                grid style=dashed,
            ]
            \addplot+[color=blue, mark=square] table[x index=0, y index=1] {\main/data/performance.dat};
            \addlegendentry{original}
            \addplot+[color=red, mark=triangle] table[x index=0, y index=2] {\main/data/performance.dat};
            \addlegendentry{rewrite}
        \end{axis}
    \end{tikzpicture}
    \label{fig:performance}
\end{figure}

From our examples we know that most of the performance is gained by rewriting the \texttt{take n arr} expression to a version that only gets its shape.
The farther we shift, the longer the original program will spend in the with-expression of this function.
This is not necessary in our rewritten program, which is why we see greatly improved performance with an increasing shift distance.
