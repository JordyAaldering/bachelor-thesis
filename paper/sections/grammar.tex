\chapter{Syntax}\label{sec:syntax}

We start this paper by creating a small array programming language, which we will use throughout this paper.
This language will have the basic functionality which most array programming languages have, with which we will be able to create most simple programs.
The syntax follows a lambda-calculus style, where a program is defined as a single expression, recursively containing sub-expressions.
Every program must evaluate to a value.

\section{Grammar}
In this section we will use the Backus-Naur form\cite{backus} to explain the program's control structure, with the added symbol `\texttt{+}' which means `one or more'.
Anything between quotes are the literal characters of the syntax.
We will also allow for comments in our language; comments start at a `\texttt{\#}', and stop at the end of a line.

Our language consists of scalars, which are simply a decimal value, and arrays.
These arrays can be made up of numbers, but they can also contain expressions.
Additionally; arrays can also be empty, meaning they do not contain any values.
We also have variable identifiers which are strings that point variable names to their values.
\begin{grammar}
<value> ::= <scalar>
     \alt <array>
     \alt <var-id>

<scalar> ::= <decimal>

<array> ::= `[]' | `[' <expr> (`,' <expr>)+ `]'
\end{grammar}

Since all our expressions evaluate to values, we can easily define binary and unary operations to transform values.
In the case of binary operations, the right expression must always be of the same shape as the left expression or it must be a scalar.
\begin{grammar}
<binary> ::= <expr> <bop> <expr>

<unary> ::= <uop> <expr>

<bop> ::= `+' | `-' | `*' | `/'                 \grammarcomment{maths}
    \alt `=' | `!=' | `>' | `>=' | `<' | `<='   \grammarcomment{equality}

<uop> ::= `-' | `||'    \grammarcomment{maths}
    \alt `!'            \grammarcomment{equality}
\end{grammar}

We can assign values to a variable by using a let-expression.
This expression takes the variable name and an expression to compute its value.
This variable will then be in scope for the second expression, after the `in'.
Similarly, we can also assign a lambda expression to an identifier, which essentially makes the identifier the name of a function which can later be called with a number of arguments.
These functions can be recursive.
\begin{grammar}
<let> ::= `let' <var-id> `=' <expr> `in' <expr>

<fun-def> ::= `let' <fun-id> `=' <lambda> `in' <expr>

<lambda> ::= ( `\\' <var-id> `.' )+ <expr>
\end{grammar}

So now we need to be able to call these user-defined functions, as well as our built-in primitive functions, by passing them values which gives us some result.
The first argument of selection is the array from which we want to select, the second argument is the index at which to select.
The selected value can be a scalar, but it can also be an array; when the dimensionality of the selection index is lower than that of the array.
We can also call a lambda function directly instead of assigning it to a function identifier first.
Every function call takes at least one, and potentially multiple, arguments.
\begin{grammar}
<call> ::= <prf>                    \grammarcomment{primitive function}
     \alt <fun-id> <expr>+          \grammarcomment{user-defined function}
     \alt `(' <lambda> `)' <expr>+  \grammarcomment{lambda application}

<prf> ::= <expr> `.(' <expr> `)'    \grammarcomment{selection}
    \alt `shape' <expr>             \grammarcomment{shape}
    \alt `dim' <expr>               \grammarcomment{dimensionality}
\end{grammar}

We also define a conditional expression.
Since our language only has scalars and arrays, and no boolean values, we define a scalar as being false if it is 0, else it is true.
One thing to note is that there there must always be an else branch.
This follows from the fact that an expression must always evaluate to a value, and thus the expression must evaluate to a value for both cases.
\begin{grammar}
<if> ::= `if' <expr> `then' <expr> `else' <expr>
\end{grammar}

Finally we have the with expression, which is arguably the most useful in array programming languages.
We use this expression to generate arrays.
This expression first takes a shape for the new value to generate, along with a default argument for the iteration.
If the shape of this argument has size $n$, that shape should correspond with the last $n$ arguments of the first expression.
Then it takes two bounds, and a variable name that will iteratively be assigned all possible values within those bounds, the dimensionality of these bounds should be the dimensionality of the first expression minus $n$.
Then the in-expression is expected to return a value of the same shape as the default value, which is then placed at the correct spot in the new array.
We also allow a shorter case, which simply assigns the default value to every element.
\begin{grammar}
<with> ::= `gen' <expr> <expr> \\
            `with' <expr> `<=' <var-id> `<' <expr> \\
            `in' <expr>
    \alt `gen' <expr> <expr>
\end{grammar}

Combining these we get the core expression, which we use to generate our programs.
\begin{grammar}
<program> ::= <expr>

<expr> ::= <fun-def>
    \alt <call>
    \alt <let>
    \alt <if>
    \alt <with>
    \alt <binary>
    \alt <unary>
    \alt <value>
\end{grammar}

\section{Example}
We will now look at part of a program as an example to see how this grammar works in practise.
This example will be used throughout the paper.
In this example we define a function \texttt{shift n arr}, this function shifts the values of the given array \texttt{n} spots to the right if \texttt{n} is positive or \texttt{n} to the left otherwise.
With zeros at all empty positions.
\begin{lstlisting}
let shift = \n.\arr.
    let pad = gen (shape (take n arr)) 0 in
    let xs = drop (-n) arr in
    if n > 0 then pad ++ xs
        else xs ++ pad
in
\end{lstlisting}

We start by defining a function \texttt{shift n arr}, which takes two arguments.
We then assign a value to the variable \texttt{pad} by using a with-expression.
This with-expression creates a new array with a shape equal to the shape of the result of the function \texttt{take n arr}, which is user-defined.
Then we assign a value to the variable xs with the other user-defined function \texttt{drop n arr}.
Then we go to the conditional expression and, depending on the value of n, we append the values of the variables pad and xs.
Here we have no more expressions so the result of this conditional expression is then the returned value when calling the function \texttt{shift n arr}.
